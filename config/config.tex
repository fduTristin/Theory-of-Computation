\usepackage{graphicx}
\usepackage{geometry}
\usepackage{hyperref}
\usepackage{amssymb}
\usepackage{booktabs}
\usepackage{tabularx}
\usepackage{amsthm}
\usepackage{amsmath}
\usepackage{enumitem}
\usepackage{tikz}
\usetikzlibrary{automata, positioning, arrows}

\geometry{left=1.2in, right=1.2in, top=1.5in, bottom=1.5in}
\linespread{1.5}%行距

% 设置列表环境的上下间距
\setenumerate[1]{itemsep=5pt,partopsep=0pt,parsep=\parskip,topsep=5pt}
\setitemize[1]{itemsep=5pt,partopsep=0pt,parsep=\parskip,topsep=5pt}
\setdescription{itemsep=5pt,partopsep=0pt,parsep=\parskip,topsep=5pt}

\theoremstyle{definition}
\newtheorem{defn}{\textbf{\textit{def}}}[section]
\newtheorem{prop}{\textbf{\textit{prop}}}[section]
\newtheorem{thm}[defn]{\textbf{\textit{thm}}}
\newtheorem{corollary}[defn]{\textbf{\textit{corollary}}}
\newtheorem{lemma}[defn]{\textit{lemma}}% 引理  准则 共用一个编号计数
\newtheorem{criterion}[defn]{\textbf{\textit{criterion}}}

\newtheorem{example}{\textbf{\textit{e.g.}}}[section]
\newtheorem{exercises}{\textbf{\textit{exercises}}}[section]
\newtheorem*{remark}{\textbf{\textit{remark}}}

\newenvironment{solution}{\par{\textit{solution}}\;}{\qed\par}

\def\R{\mathbb{R}} % 实数域
\def\N{\mathbb{N}} % 自然数域
\def\Q{\mathbb{Q}} % 有理数域
\def\eps{\varepsilon} %ε

\def\bgtbl{\begin{table}[htbp]
    \centering
    \begin{tabularx}{\textwidth}{XXXXX}}
\def\bgpic{\begin{tikzpicture}[->,>=stealth',shorten >=1pt,auto,node distance=2.5cm,semithick]
    \tikzstyle{every state}=[fill=white,draw=black,text=black]}
\graphicspath{ {pictures/},{../pictures/}}  % 配置图形文件检索目录